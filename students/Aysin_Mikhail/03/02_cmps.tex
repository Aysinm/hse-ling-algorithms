\documentclass[12pt]{extarticle}
\usepackage{geometry,nopageno}
\geometry{a5paper,left=1cm,right=1cm,top=1cm,bottom=1cm}
\usepackage{cmap, type1ec}
\usepackage[T2A]{fontenc}
\usepackage[utf8]{inputenc}
\usepackage[russian]{babel}

\usepackage{verbatim,nameref}
\usepackage{amsmath,amsthm,amstext,amssymb,amscd,
            mathtools,mathrsfs,dsfont}
            
\newtheorem*{problem}{Задача}

\begin{document}

\begin{problem}[2]
{\em В доказательстве нижней границы сложности сортировок на основе сравнений фигурировало выражение $\log_2{n!}$, которое мы оценивали формулой Стирлинга. Но можно ограничиться и школьной математикой: используя лишь свойство логарифма произведения, докажите более слабое утверждение: $\log_2{n!} \geqslant C\,n\log_2{n}$?}
\end{problem}

\begin{proof}
        
По свойству логарифма произведения 
$$\log_a{(x\cdot y)} = \log_a{x} + \log_a{y}, \;\;\forall x, y > 0$$

$$ \log_2{n!} = \log_2{(n\cdot (n-1)\cdot (n-2) \cdot \ldots \cdot \frac{n}{2} \cdot \ldots \cdot 2 \cdot 1)} = $$
$$   = \log_2{n} + \log_2{(n-1)}+ \log_2{(n-2)} + \ldots + \log_2{\frac{n}{2}} +\ldots + \log_2{2} + \log_2{1}  \geqslant $$
$$   \geqslant  \log_2{\frac{n}{2}} + \log_2{\frac{n}{2}}+ \log_2{\frac{n}{2}} + \ldots + \log_2{\frac{n}{2}} + 0 + \ldots + 0 + 0 = $$
$$ = \frac{n}{2} \log_2{\frac{n}{2}} = O (n \log n)$$

\end{proof}

\end{document}

